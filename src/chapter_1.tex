\chapter{Electricidad}

\section{Fuerza eléctrica}

Según el modelo atómico, la materia está formada por moléculas que están compuestas en última instancia por átomos.
Pero a su vez, los átomos están formados por distintas configuraciones de \emph{partículas subatómicas}, de ahí que haya distintos tipos de elementos.
Cada elemento se clasifica en la tabla periódica según su número atómico, que es la cantidad de protones que tiene un átomo de dicho elemento.
Un átomo tiene la misma cantidad de protones que de neutrones, pero no necesariamente la misma cantidad de electrones.
Podemos identificar así el peso de un átomo de cierto elemento, conociendo la cantidad de partículas subatómicas, según el siguiente cuadro, donde además se observa la carga eléctrica de cada una.

\begin{table}[h!]
    \begin{center}
        \begin{tabular}{|c|c|c|}
            \hline
            Partícula & Masa [\si{\kilo\gram}] & Carga [\si{\coulomb}]
            \\ \hline \hline
            Electrón & $9.1094 \times 10^{-31}$ & $-1.6022 \times 10^{-19}$
            \\ \hline
            Protón & $1.6726 \times 10^{-27}$ & $+1.6022 \times 10^{-19}$
            \\ \hline
            Neutrón & $1.6749 \times 10^{-27}$ & $0$
            \\ \hline
        \end{tabular}
        \caption{Masa y carga de las partículas atómicas.}
    \end{center}
\end{table}

La cantidad de protones (y neutrones) de un átomo permanece inalterada.
Una transformación física que modificase la \emph{estructura atómica} implicaría un cambio en el núcleo del átomo haciendo que el elemento no sea el mismo.
Estas transformaciones se conocen como fusión y fisión nuclear.

En cambio, la cantidad de electrones de un átomo sí puede variar sin alterar la naturaleza del elemento.
Si un átomo tiene la misma cantidad de protones que de electrones se lo llama neutro.
De lo contrario, se lo llama ion, o se dice que está cargado eléctricamente.

\concept{Aislantes (o dieléctricos) y conductores:}

Un dieléctrico es un material que no es conductor de la electricidad.
Para el vacío, se definen la constante de Coulomb $(\cstcoulomb)$ y la permitividad del vacío $(\epsilon_0)$.
\begin{gather*}
    \cstcoulomb=8.99 \times 10^9\,\si{\newton\metre^2\per\coulomb^2}
    \\
    \epsilon_0 = \dfrac{1}{4 \pi \cstcoulomb} = 8.85 \times 10^{-12} \,\si{\coulomb^2\per\newton\metre^2}
\end{gather*}

Según las propiedades de cada dieléctrico se puede definir una constante dieléctrica $k$ y una resistencia dieléctrica $\epsilon$ para cada material en función del paso de corriente que permitan.

En un conductor, en cambio, los electrones de los átomos que lo componen pueden moverse por los orbitales externos de cada átomo, permitiendo el paso de cargas o corriente.

Tanto los aislantes como los conductores pueden tener carga.
En caso de un aislante tener carga, esta se puede localizar en algunos de los átomos que lo componen, mientras que otros de sus átomos pueden tener carga menor o neutra.

\concept{Polarización:}

Los electrones de un objeto conductor se reacomodan internamente ante la presencia de una carga que se encuentre en su cercanía sin estar en contacto.
Esto genera que, mientras el objeto externo cargado esté presente, una mitad del conductor originalmente neutro queda con carga negativa y la otra con carga positiva, ambas equivalentes a la carga externa presente.
Si el objeto externo con carga se quita, los electrones del conductor se reacomodan para devolverle su carga neutra, ya que las cargas polarizadas son del signo contrario y se atraen entre sí.

\begin{itemize}
    \item \textbf{Carga por conducción}
    
    \begin{itemize}
        \item El vidrio pierde electrones al ser frotado con seda.
        \item La goma gana electrones al ser frotada con la piel.
        \item Se dice de el vidrio queda cargado positivamente por tener más protones que electrones.
        \item Se dice que la goma queda cargada negativamente por tener más electrones que protones.
    \end{itemize}

    \item \textbf{Carga por inducción}
    
    \begin{itemize}
        \item Se acerca un objeto cargado a uno neutro que está conectado a tierra.
        \item Las cargas son empujadas y se ``fugan'' por la conexión a tierra.
        \item Se corta la conexión a tierra y se aleja el objeto cargado.
        \item El objeto, antes neutro, perdió las cargas empujadas quedando cargado.
    \end{itemize}

    \begin{center}
        \def\svgwidth{\linewidth}
        \input{./images/elec-conduc-induc.pdf_tex}
    \end{center}
\end{itemize}

Los objetos cargados se repelen o atraen entre sí por acción de una fuerza.
La fuerza eléctrica entre dos cargas puntuales está dada por la \emph{Ley de Coulomb} a continuación.

\begin{mdframed}[style=MyFrame1]
    \begin{defn}
    \end{defn}
    \cusTi{Ley de Coulomb}
    \begin{equation*}
        \Vec{F}_e = \cstcoulomb \, \frac{ q_0 \, q_1}{r^2} \, \versor{r}
    \end{equation*}
\end{mdframed}


\section{Campo eléctrico}

\begin{mdframed}[style=MyFrame1]
    \begin{defn}
    \end{defn}
    \cusTi{Campo eléctrico de carga puntual}
    \cusTe{Campo vectorial generado por un cuerpo con carga que indica la fuerza eléctrica que sufriría cierta carga $q_0$ en un punto del espacio.}
    \begin{equation*}
        \Vec{E} = \frac{\Vec{F}_e}{q_0}
    \end{equation*}
\end{mdframed}

Según el principio de superposición, el campo eléctrico generado por $\nth$ cargas puntuales, va a estar dado por la suma de la contribución que cada carga genere:
\begin{equation}
    \Vec{E}= \cstcoulomb \sum_{\ith=1}^\nth \frac{q_\ith}{r_\ith^2} \, \versor{r}_\ith
\end{equation}

Ahora bien, el campo eléctrico generado por una carga contínua se define a partir del diferencial $\dif q$ que determina la distribución geométrica de la carga.

Según esté la carga dispuesta en una, dos o tres dimensiones se tendrá una densidad de carga lineal, superficial o volumétrica respectivamente.
Por lo tanto $\dif q$ será, en cada caso:
\begin{align*}
    \dif q &= \lambda \, \dif x
    \\[1ex]
    \dif q &= \sigma \, \dif A
    \\[1ex]
    \dif q &= \rho \, \dif V
\end{align*}

Y si la carga está distribuída uniformemente, la densidad de carga es igual a la carga total $Q$ sobre el largo, área o volumen del material:
\begin{align*}
    \lambda &= \frac{Q}{L}
    \\[1ex]
    \sigma &= \frac{Q}{A}
    \\[1ex]
    \rho &= \frac{Q}{V}
\end{align*}

Se consideran particiones que van a tener cierta carga $\Delta q_\ith$ y se las trata como puntuales.
Sumando el aporte de cada $\Delta q$ se obtiene una aproximación del campo eléctrico debido a la carga total:
\begin{equation*}
    \Vec{E} \approx \cstcoulomb \sum_{\ith=1}^\nth \frac{\Delta q_\ith}{r_\ith^2} \, \versor{r}_\ith
\end{equation*}

Luego, tomar el límite cuando $\nth\to\infty$ en la ecuación anterior equivale a hacer infinito el número de particiones.
De esta forma, el aporte de cada trozo de carga será $\Delta q_\ith = \dif q$ pudiendo definir:

\begin{mdframed}[style=MyFrame1]
    \begin{defn}
    \end{defn}
    \cusTi{Campo eléctrico de carga contínua}
    \begin{equation*}
        \Vec{E} = \cstcoulomb \int \frac{\dif q}{r^2} \, \versor{r}
    \end{equation*}
\end{mdframed}

Si bien la definición es siempre válida, la integral solo se puede computar en aquellos puntos que sea posible identificar:
\begin{itemize}
    \item Una simetría que permita escribir $\versor{r}$ como uno de los versores canónicos.
    \item Una distancia $r$ tal que $\dif q$ varíe en una sola dimensión.
\end{itemize}

\begin{mdframed}[style=MyFrame2]
    \begin{example}
    \end{example}
    \cusTi{Campo eléctrico de una barra}
    \begin{formatI}
        Calcular el campo eléctrico en el punto $\vec{x}_0$ generado por una barra aislante con densidad de carga lineal uniforme.
        Suponer que la barra tiene grosor nulo.
    \end{formatI}

    \begin{center}
        \def\svgwidth{\linewidth}
        \input{./images/elec-barra.pdf_tex}
    \end{center}

    Observar que $r$ varía linealmente a medida que se recorre la barra.
    \begin{equation*}
        \vec{E} = \cstcoulomb \int_{d}^{d+l} \frac{\lambda \, \dif x}{x^2} \inParentheses{- \iVer}
    \end{equation*}
\end{mdframed}

\begin{mdframed}[style=MyFrame2]
    \begin{example}
    \end{example}
    \cusTi{Campo eléctrico de un anillo}
    \begin{formatI}
        Calcular el campo eléctrico en el punto $\vec{x}_0$ generado por un anillo aislante con densidad de carga lineal uniforme.
        Suponer que el anillo tiene grosor nulo.
    \end{formatI}

    \begin{center}
        \def\svgwidth{\linewidth}
        \input{./images/elec-anillo.pdf_tex}
    \end{center}

    Dado que $r$ no varía,
    \begin{equation*}
        \vec{E} = \cstcoulomb \, \frac{\versor{r}}{r^2} \int \dif q
    \end{equation*}
    
    Donde:
    \begin{equation*}
        \left\{
        \begin{aligned}
            & \int \dif q = Q
            \\
            & r = \sqrt{D^2 + R_0^2}
            \\
            & \versor{r} = \nnorm{\versor{r}} \inBrackets{\cos(\theta) \, \iVer + \sin(\theta) \, \jVer}
        \end{aligned}
        \right.
    \end{equation*}

    Además, por geometría y simetría:
    \begin{equation*}
        \cos(\theta) = \frac{D}{r} \implies \versor{r} = \frac{D}{r} \, \iVer
    \end{equation*}

    Obteniendo
    \begin{equation*}
        \vec{E} = \frac{\cstcoulomb \, D \, Q}{\inParentheses{D^2 + R_0^2}^{\frac{3}{2}}} \, \iVer
    \end{equation*}
\end{mdframed}

\begin{mdframed}[style=MyFrame2]
    \begin{example}
    \end{example}
    \cusTi{Campo eléctrico de un disco}
    \begin{formatI}
        Calcular el campo eléctrico en el punto $\vec{x}_0$ generado por un disco aislante con densidad de carga superficial uniforme.
    \end{formatI}

    \begin{center}
        \def\svgwidth{\linewidth}
        \input{./images/elec-disco.pdf_tex}
    \end{center}

    Observar que $r$ varía axialmente a medida que se recorre el disco.
    \begin{equation*}
        \vec{E} = \cstcoulomb \iint \frac{\sigma \, \dif A}{r^2} \, \versor{r}
    \end{equation*}

    Cambiando a coordenadas polares:
    \begin{gather*}
        \fx{T}{R,\phi} = (y,z) = \inBrackets{R \cos(\phi) &,& R \sin(\phi)}
        \\
        \det \inParentheses{\fx{J}{T}} = \cos(\phi) \, R \cos(\phi) - \sin(\phi) \inParentheses{-R \sin(\phi)}
        \\
        \det \inParentheses{\fx{J}{T}} = R \implies
        \\
        \dif A = R \, \dif \phi \, \dif R
    \end{gather*}

    Se puede reescribir la integral de campo eléctrico como sigue:
    \begin{equation*}
        \vec{E} = \cstcoulomb \iint \frac{\sigma \, R \, \dif \phi \, \dif R}{r^2} \, \versor{r}
    \end{equation*}

    Donde:
    \begin{equation*}
        \left\{
        \begin{aligned}
            & r = \sqrt{D^2 + R^2}
            \\
            & \versor{r} = \nnorm{\versor{r}} \inBrackets{\cos(\theta) \, \iVer + \sin(\theta) \, \jVer}
        \end{aligned}
        \right.
    \end{equation*}

    Además, por geometría y simetría:
    \begin{equation*}
        \cos(\theta) = \frac{D}{r} \implies \versor{r} = \frac{D}{r} \, \iVer
    \end{equation*}
    
    Reemplazando $\versor{r}$ y $r$ nuevamente se obtiene:
    \begin{align*}
        \vec{E} &= \cstcoulomb \, \sigma \, D \iint \frac{R \, \dif \phi \, \dif R}{\inParentheses{D^2 + R^2}^{\frac{3}{2}}} \, \iVer
        \\[1ex]
        &= \cstcoulomb \, \sigma \, D \int_0^{2\pi} \dif \phi \, \int_0^{R_0} \frac{R \, \dif R}{\inParentheses{D^2 + R^2}^{\frac{3}{2}}} \, \iVer
        \\[1ex]
        &= 2 \, \pi \, \cstcoulomb \, \sigma \, D \int_0^{R_0} \frac{R \, \dif R}{\inParentheses{D^2 + R^2}^{\frac{3}{2}}} \, \iVer
    \end{align*}
\end{mdframed}

\begin{mdframed}[style=MyFrame2]
    \begin{example}
        \label{eg:elec-placa}
    \end{example}
    \cusTi{Campo eléctrico de una placa infinita}
    \begin{formatI}
        Calcular el campo eléctrico en el punto $\vec{x}_0$ generado por una placa infinita aislante con densidad de carga superficial uniforme.
    \end{formatI}
    \begin{center}
        \def\svgwidth{\linewidth}
        \input{./images/elec-placa.pdf_tex}
    \end{center}
    Una placa infinita es equivalente a un disco infinito.
    Es posible hacer el mismo razonamiento para el disco y hacer tender el radio a infinito.
    Considerando $R_0 \to \infty$ en el resultado obtenido anteriormente:
    \begin{equation*}
        \vec{E} = 2 \, \pi \, \cstcoulomb \, \sigma \, D \int_0^\infty \frac{R \, \dif R}{\inParentheses{D^2 + R^2}^{\frac{3}{2}}} \, \iVer
    \end{equation*}
    Sea $u = D^2 + R^2$ tal que si $R \to \infty \implies u \to \infty$ y
    \begin{equation*}
        \frac{\dif u}{\dif R} = 2 \, R \implies \dif R = \frac{\dif u}{2 \, R}
    \end{equation*}

    De manera que la integral de campo es
    \begin{align*}
        \vec{E} &= \pi \, \cstcoulomb \, \sigma \, D \int_0^\infty \frac{\dif u}{u^{\frac{3}{2}}} \, \iVer
        \\[1ex]
        &= \pi \, \cstcoulomb \, \sigma \, D \, \barrow{\frac{1}{-\frac{1}{2} \sqrt{u}}}{u=0}{u=\infty}
        \\[1ex]
        &= 2 \, \pi \, \cstcoulomb \, \sigma \, D \, \barrow{\frac{1}{\sqrt{D^2 + R^2}}}{R=\infty}{R=0}
        \\
        &= 2 \, \pi \, \cstcoulomb \, \sigma
        \\
        &= \frac{\sigma}{2 \, \epsilon_0}
    \end{align*}

    Observar que el resultado es independiente de $D$.
    Esto implica que la dirección y magnitud del campo es constante para todo $\vec{x}$.
    
    Este resultado se verificará mediante otro método en el ejemplo \ref{eg:gauss-placa}.
\end{mdframed}

\begin{mdframed}[style=MyFrame2]
    \begin{example}
    \end{example}
    \cusTi{Campo eléctrico de un cilindro}
    \begin{formatI}
        Calcular el campo eléctrico en el punto $\vec{x}_0$ generado por un cilindro aislante con densidad de carga superficial uniforme.
        Considerar el caso en que el cilindro es hueco y el caso en que es sólido.
    \end{formatI}
    \begin{center}
        \def\svgwidth{\linewidth}
        \input{./images/elec-cilindro.pdf_tex}
    \end{center}

    El campo eléctrico para una carga contínua es
    \begin{equation*}
        \vec{E} = \cstcoulomb \int \frac{\dif q}{r^2} \, \versor{r}
    \end{equation*}

    \concept{Caso de cilindro hueco (sin tapas):}

    El diferencial de carga está dado por
    \begin{equation*}
        \dif q = \sigma \, \dif A
        = \sigma \, \dif s \, \dif x
        = \sigma \, R_0 \, \dif \phi \, \dif x
    \end{equation*}

    Por simetría axial el versor director es
    \begin{equation*}
        \versor{r} = \cos(\theta) \, \iVer
        = \frac{L+D-x}{r} \, \iVer
    \end{equation*}

    Y la distancia $r$ es
    \begin{equation*}
        r = \sqrt{R_0^2 + \inParentheses{L+D-x}^2}
    \end{equation*}

    Con lo cual, la integral de campo eléctrico queda
    \begin{align*}
        \vec{E} &= \cstcoulomb \iint \frac{\sigma \, R_0 \, \dif \phi \, \dif x}{r^2} \, \frac{L+D-x}{r} \, \iVer
        \\[1em]
        &= \cstcoulomb \, \sigma \, R_0 \iint \frac{L+D-x}{r^3} \dif \phi \, \dif x \, \iVer
        \\[1em]
        &= 2 \, \pi \, \cstcoulomb \, \sigma \, R_0 \int_0^L \frac{L+D-x}{\inBrackets{R_0^2 + \inParentheses{L+D-x}^2}^{\frac{3}{2}}} \dif x \, \iVer
    \end{align*}

    \concept{Caso de cilindro sólido:}

    El diferencial de carga está dado por
    \begin{equation*}
        \dif q = \rho \, \dif V
        = \rho \, \pi \, R_0^2 \, \dif x
    \end{equation*}

    Por simetría axial el versor director es
    \begin{equation*}
        \versor{r} = \cos(\theta) \, \iVer
        = \frac{L+D-x}{r} \, \iVer
    \end{equation*}

    Y la distancia $r$ es
    \begin{equation*}
        r = \sqrt{R_0^2 + \inParentheses{L+D-x}^2}
    \end{equation*}

    Con lo cual, la integral de campo eléctrico queda
    \begin{align*}
        \vec{E} &= \cstcoulomb \int \frac{\rho \, \pi \, R_0^2 \, \dif x}{r^2} \, \frac{L+D-x}{r} \, \iVer
        \\[1em]
        &= \pi \, \cstcoulomb \, \rho \, R_0^2 \int \frac{L+D-x}{r^3} \dif x \, \iVer
        \\[1em]
        &= \pi \, \cstcoulomb \, \rho \, R_0^2 \int_0^L \frac{L+D-x}{\inBrackets{R_0^2 + \inParentheses{L+D-x}^2}^{\frac{3}{2}}} \dif x \, \iVer
    \end{align*}
\end{mdframed}

\begin{mdframed}[style=MyFrame2]
    \begin{example}
    \end{example}
    \cusTi{Campo eléctrico de un cono}
    \begin{formatI}
        Calcular el campo eléctrico en el punto $\vec{x}_0$ generado por un cono aislante con densidad de carga superficial uniforme.
        Considerar el caso en que el cono es hueco y el caso en que es sólido.
    \end{formatI}

    \begin{center}
        \def\svgwidth{\linewidth}
        \input{./images/elec-cono.pdf_tex}
    \end{center}

    El campo eléctrico para una carga contínua es
    \begin{equation*}
        \vec{E} = \cstcoulomb \int \frac{\dif q}{r^2} \, \versor{r}
    \end{equation*}

    \concept{Caso de cono hueco (sin tapa):}

    El diferencial de carga está dado por
    \begin{equation*}
        \dif q = \sigma \, \dif A
        = \sigma \, \dif s \, \dif x
        = \sigma \, \fx{R}{x} \, \dif \phi \, \dif x
    \end{equation*}

    Por simetría axial el versor director es
    \begin{equation*}
        \versor{r} = \cos(\theta) \, \iVer
        = \frac{D-x}{r} \, \iVer
    \end{equation*}

    Y la distancia $r$ es
    \begin{equation*}
        r = \sqrt{\inParentheses{D-x}^2 + \inBrackets{\fx{R}{x}}^2}
    \end{equation*}

    Al tratarse de la pendiente de la recta del cono en corte $y=0$, el radio varía según
    \begin{equation*}
        \fx{R}{x} = \frac{R_0}{L} \, x
    \end{equation*}

    Con lo cual, la integral de campo eléctrico queda
    \begin{align*}
        \vec{E} &= \cstcoulomb \iint \frac{\sigma \, \fx{R}{x} \, \dif \phi \, \dif x}{\inParentheses{D-x}^2 + \inBrackets{\fx{R}{x}}^2} \, \frac{D-x}{r} \, \iVer
        \\[1em]
        &= \cstcoulomb \, \sigma \, \frac{R_0}{L} \iint \frac{x \inParentheses{D-x}}{\inBrackets{\inParentheses{D-x}^2 + \inParentheses{\frac{R_0}{L} \, x}^2}^{\frac{3}{2}}} \, \dif \phi \, \dif x \, \iVer
        \\[1em]
        &= 2 \, \pi \, \cstcoulomb \, \sigma \, \frac{R_0}{L} \int_0^L \frac{x \inParentheses{D-x}}{\inBrackets{\inParentheses{D-x}^2 + \inParentheses{\frac{R_0}{L} \, x}^2}^{\frac{3}{2}}} \, \dif x \, \iVer
    \end{align*}

    \concept{Caso de cono sólido:}

    \begin{equation*}
        \dif q = \rho \, \dif V = \rho \, \pi \, \inBrackets{\fx{R}{x}}^2 \, \dif x
    \end{equation*}

    Por simetría axial el versor director es
    \begin{equation*}
        \versor{r} = \cos(\theta) \, \iVer
        = \frac{D-x}{r} \, \iVer
    \end{equation*}

    Y la distancia $r$ es
    \begin{equation*}
        r = \sqrt{\inParentheses{D-x}^2 + \inBrackets{\fx{R}{x}}^2}
    \end{equation*}

    Al tratarse de la pendiente de la recta del cono en corte $y=0$, el radio varía según
    \begin{equation*}
        \fx{R}{x} = \frac{R_0}{L} \, x
    \end{equation*}

    Con lo cual, la integral de campo eléctrico queda
    \begin{align*}
        \vec{E} &= \cstcoulomb \int \frac{\rho \, \pi \, \inBrackets{\fx{R}{x}}^2 \, \dif x}{r^2} \, \frac{D-x}{r} \, \iVer
        \\[1em]
        &= \pi \, \cstcoulomb \, \rho \int \frac{\inBrackets{\fx{R}{x}}^2 \inParentheses{D-x}}{r^3} \, \dif x \, \iVer
        \\[1em]
        &= \pi \, \cstcoulomb \, \rho \inParentheses{\frac{R_0}{L}}^2 \int_0^L \frac{x^2 \inParentheses{D-x}}{\inBrackets{\inParentheses{D-x}^2 + \inParentheses{\frac{R_0}{L} \, x}^2}^{\frac{3}{2}}} \, \dif x \, \iVer
    \end{align*}
\end{mdframed}


\section{Ley de Gauss}

El flujo eléctrico ($\Phi$) de una carga eléctrica encerrada por una superficie ($S$) está dado por el campo eléctrico ($\Vec{E}$) según
\begin{equation*}
    \Phi = \iint_S \Vec{E} \cdot \dif \Vec{S} = \iint \Vec{E} \inParentheses{\Vec{s}(u,v)} \versor{n} \, \dif S
\end{equation*}

La Ley de Gauss establece que el flujo no depende de la ubicación de la carga dentro de una superficie cerrada
Esto se debe a que el campo eléctrico es inversamente proporcional a la distancia mientras que el área es directamente proporcional.
Así, podemos reducir cualquier caso de estudio a una superficie esférica con la carga ubicada en el centro:
\begin{equation*}
    \Phi = E \, A = \frac{\cstcoulomb \, \sub{q}{in}}{r^2} \, 4 \, \pi \, r^2 = 4 \, \pi \, \cstcoulomb \, \sub{q}{in} = \frac{\sub{q}{in}}{\epsilon_0}
\end{equation*}

Si el campo eléctrico tiene magnitud $E$ constante y dirección normal a una superficie de área $A$ entonces se puede calcular la magnitud del campo en un punto de la superficie.
\begin{equation*}
    \Phi = E \, \oiint_S \dif S = E \, A
\end{equation*}

\begin{mdframed}[style=MyFrame2]
    \begin{example}
    \end{example}
    \cusTi{Simetría cilíndrica}
    \begin{formatI}
        Calcular el campo eléctrico en el punto $\vec{x}_0$ generado por una barra infinita aislante con densidad de carga lineal uniforme.
    \end{formatI}
    \begin{center}
        \def\svgwidth{0.6\linewidth}
        \input{./images/gauss-barra.pdf_tex}
    \end{center}
    \begin{equation*}
        \left\{
        \begin{aligned}
            \Phi &= E \, A_2 = E \, 2 \, \pi \, r \, L
            \\
            \Phi &= \frac{\sub{q}{in}}{\epsilon_0} = \frac{\lambda \, L}{\epsilon_0}
        \end{aligned}
        \right.
    \end{equation*}
    Luego:
    \begin{equation*}
        E = \frac{\lambda}{2 \, \pi r \, \epsilon_0}
    \end{equation*}
\end{mdframed}

\begin{mdframed}[style=MyFrame2]
    \begin{example}
        \label{eg:gauss-placa}
    \end{example}
    \cusTi{Simetría bilateral}
    \begin{formatI}
        Calcular el campo eléctrico en el punto $\vec{x}_0$ generado por una placa infinita aislante con densidad de carga superficial uniforme.
    \end{formatI}
    \begin{center}
        \def\svgwidth{\linewidth}
        \input{./images/gauss-placa.pdf_tex}
    \end{center}
    \begin{equation*}
        \left\{
        \begin{aligned}
            \Phi &= E \, \sub{A}{tot} = E \inParentheses{A_1+A_2} = 2 \, E \, A
            \\
            \Phi &= \frac{\sub{q}{in}}{\epsilon_0} = \frac{\sigma \, A}{\epsilon_0}
        \end{aligned}
        \right.
    \end{equation*}
    Luego:
    \begin{equation*}
        E = \frac{\sigma}{2 \, \epsilon_0}
    \end{equation*}
    Obteniendo el mismo resultado que en el ejemplo \ref{eg:elec-placa} independientemente de $D$.
\end{mdframed}

\begin{mdframed}[style=MyFrame2]
    \begin{example}
    \end{example}
    \cusTi{Simetría esférica}
    \begin{formatI}
        Calcular el campo eléctrico en el punto $\vec{x}_0$ generado por una esfera aislante de radio $R$ que tiene una densidad de carga volumétrica uniforme.
    \end{formatI}
    \begin{center}
        \def\svgwidth{0.7\linewidth}
        \input{./images/gauss-esfera.pdf_tex}
    \end{center}
    \begin{equation*}
        \left\{
        \begin{aligned}
            \Phi &= E \, A = E \, 4 \pi \, r^2
            \\
            \Phi &= \frac{\sub{q}{in}}{\epsilon_0}
        \end{aligned}
        \right.
    \end{equation*}
    Luego:
    \begin{equation*}
        E = \frac{\sub{q}{in}}{4 \, \pi \, \epsilon_0 \, r^2}
    \end{equation*}
    Según el volumen de carga encerrado $V$ por la superficie gaussiana, se van a obtener distintos valores para la magnitud del campo.
    La carga encerrada depende del radio $r$, ya que está dada por
    \begin{equation*}
        \fx{\sub{q}{in}}{r} = \rho \, \fx{V}{r}
    \end{equation*}
    Pero el volumen encerrado no puede ser mayor que $\fx{V}{R}$ pudiendo distinguir dos casos:
    \begin{equation*}
        \left\{
        \begin{aligned}
            r<R & \implies q_1 = \rho \, \fx{V}{r} = \frac{Q}{\frac{4 \, \pi \, R^3}{3}} \, \frac{4 \, \pi \, r^3}{3} = \frac{Q \, r^3}{R^3}
            \\
            r>R & \implies q_2 = \rho \, \fx{V}{R} = \frac{Q}{\frac{4 \, \pi \, R^3}{3}} \, \frac{4 \, \pi \, R^3}{3} = Q
        \end{aligned}
        \right.
    \end{equation*}
    Obteniendo las siguientes expresiones para el campo eléctrico al reemplazar $q_1$ o $q_2$:
        \begin{equation}
            r<R \implies E = \frac{Q \, r}{4 \, \pi \, \epsilon_0 \, R^3}
            \label{eqn:esferaInt}
        \end{equation}
        \begin{equation}
            r>R \implies E = \frac{Q}{4 \, \pi \, \epsilon_0 \, r^2}
            \label{eqn:esferaExt}
        \end{equation}
    \begin{center}
        \def\svgwidth{0.7\linewidth}
        \input{./images/gauss-esfera-2.pdf_tex}
    \end{center}
\end{mdframed}


\subsection{Equilibrio electrostático}

Se dice que un conductor esta en equilibrio electrostático cuando no tiene electrones en movimiento.

En otras palabras, se trata de un conductor por el que no pasa corriente.
Los conductores que cumplan esta condición, tienen las siguientes propiedades:

\begin{itemize}
\item El campo eléctrico es nulo en el interior del conductor, ya sea este hueco o sólido.

\item Si es un conductor con carga, está se acumula en la superficie.

\item El campo es ortogonal a la superficie del conductor.

\item La densidad de carga superficial es máxima donde la curvatura sea mínima.
\end{itemize}

\begin{mdframed}[style=MyFrame2]
    \begin{example}
    \end{example}
    \begin{formatI}
        Se tiene una esfera aislante de radio $r_0$ con carga $q_1 = Q$ que se encuenta en el centro de un cascarón conductor de ancho $\Delta r = r_2 - r_1$ que tiene $q_2 = -2\,Q$ de carga.
        Se quiere calcular el campo eléctrico en todo el espacio en función de la distancia al centro de la esfera.
    \end{formatI}
    \begin{center}
        \def\svgwidth{0.7\linewidth}
        \input{./images/eq-electrostatico.pdf_tex}
    \end{center}
    \begin{itemize}
        \item Para $r<r_0$ el cascarón no afecta, por lo que el campo está dado según la ecuación \ref{eqn:esferaInt}.
        \item Para $r_0<r<r_1$ el cascarón no afecta, por lo que el campo está dado según la ecuación \ref{eqn:esferaExt}.
        \item Para $r_1<r<r_2$ el campo eléctrico es nulo por tratarse del interior de un conductor.
        \item Para $r_2<r$ el campo es $E = \frac{-Q}{4 \, \pi \, \epsilon_0 \, r^2}$ dado que la carga neta que encierra una superficie gaussiana es $\sub{q}{in} = q_1 + q_2 = -Q$.
    \end{itemize}
\end{mdframed}


\section{Potencial eléctrico}

Si dos cargas son del mismo signo, el medio tiene que hacer trabajo para acercarlas, ganando el sistema energía potencial eléctrica.
Si dos cargas son de signo opuesto, el sistema gana energía potencial cuando estas son alejadas, ya que el medio tiene que hacer trabajo para alejarlas.
Así, se establece por convención que el trabajo es positivo cuando el sistema gana energía.
\begin{equation*}
    W \gtrless 0 \iff {\sub{E}{pot}}_F \gtrless {\sub{E}{pot}}_0 \iff \Delta \sub{E}{pot} \gtrless 0
\end{equation*}

Se tiene una carga $q_1 > 0$ fija en un punto.
Se coloca una segunda carga $q_2 < 0$ a una distancia infinitamente grande
En principio $q_2$ no se vería afectada por el campo eléctrico de $q_1$, ya que en el infinito sería nulo.
Ahora bien, si se la perturba de la posición inicial, partiría con cierta velocidad inicial para acelerarse en dirección hacia $q_1$.
La fuerza causante de esta aceleración sería la fuerza eléctrica ejercida por $q_1$.

No obstante, se aplica sobre $q_2$ una fuerza externa de igual magnitud y sentido contrario a la fuerza eléctrica.
Así, $q_2$ se mueve a velocidad constante en dirección hacia $q_1$ tal que $\Delta \sub{E}{cin} = 0$.

El trabajo de la fuerza externa cuantifica la energía que tiene $q_2$ por estar en presencia del campo producido por $q_1$, energía que potencialmente podría convertirse en energía cinética si no hubiese fuerza externa.
Pero el trabajo de la fuerza externa (no conservativa) es igual a menos el trabajo de la fuerza eléctrica (conservativa).
Esto es, renombrando a $q_2$ como $q_0$:
\begin{align*}
    \Delta \sub{E}{pot} &= - \sub{W}{con}
    \\
    &= - \int_C \sub{\Vec{F}}{ele} \cdot \dif \Vec{s}
    \\
    &= - q_0 \int_C \Vec{E} \cdot \dif \Vec{s}
\end{align*}

Dado que $q_0 < 0$ y $-\sub{W}{con} = \sub{W}{no con} < 0$ es posible mantener el menos en la ecuación y trabajar con valores absolutos ya que los negativos se cancelan entre sí.
Es decir, si se hubiese tratado de una carga de prueba positiva, el trabajo de la fuerza externa para llevar $q_2$ en dirección hacia $q_1$ hubiese sido positivo también.

Finalmente, se define la diferencia de potencial eléctrico $(\Delta v)$ dividiendo por $q_0$ ambos miembros.

\begin{mdframed}[style=MyFrame1]
    \begin{defn}
        \label{defn:potEle}
    \end{defn}
    \cusTi{Potencial eléctrico}
    \begin{equation*}
        \Delta \voltage = - \int_C \vec{E} \cdot \dif \vec{s}
    \end{equation*}
\end{mdframed}

A partir de la definición anterior, se deduce el potencial eléctrico para una carga puntual:
\begin{align*}
    \Delta \voltage &= - \int_C \frac{\cstcoulomb \, q}{r^2} \versor{r} \cdot \dif \Vec{s}
    \\
    &= - \int_C \frac{\cstcoulomb \, q}{r^2} \nnorm{\versor{r}} \, \nnorm{\dif \Vec{s}} \, \cos(\theta)
    \\
    &= - \int_{r_0}^{r_1} \frac{\cstcoulomb \, q}{r^2} \dif r
    \\
    &= \barrow{\frac{\cstcoulomb \, q}{r}}{r_0}{r_1}
\end{align*}

Pudiendo luego generalizar para $N$ cargas puntuales:
\begin{equation}
    \Delta \voltage = \cstcoulomb \sum_{\ith=1}^\nth \frac{q_\ith}{r_\ith}
\end{equation}

Y tomando el límite cuando $\nth\to\infty$ en la ecuación anterior, se define el potencial en un punto del espacio dado por una carga contínua.

\begin{mdframed}[style=MyFrame1]
    \begin{defn}
        \label{defn:potCargaCont}
    \end{defn}
    \cusTi{Potencial eléctrico de carga contínua}
    \begin{equation*}
        \Delta \voltage = \cstcoulomb \int \frac{\dif q}{r}
    \end{equation*}
\end{mdframed}

La definición \ref{defn:potEle} está dada para cargas puntuales.
No confundir la integral involucrada con la integral de la definición \ref{defn:potCargaCont}.
La primera implica una integral de tipo 2 para el trabajo, mientras que la segunda es una integral de Riemann para sumar el aporte de infinitas cargas puntuales.

\begin{mdframed}[style=MyFrame2]
    \begin{example}
    \end{example}
    \cusTi{Potencial en campo uniforme}
    \cusTe{Se quiere calcular el potencial eléctrico o bien entre dos placas paralelas con carga opuesta o bien para una única placa infinita con cierta carga.}
    Partiendo de la definición \ref{defn:potEle} se calcula el potencial.
    Esto es, la energía cinética que ganaría una carga de prueba en ir desde una placa hacia la otra, o en recorrer cierta distancia $\Delta x$ por la fuerza eléctrica ejercida por una placa infinita.
    \begin{align*}
        \Delta \voltage &= - E \int_{x_0}^{x_1} \dif x
        \\
        &= - E \, \Delta x
    \end{align*}
\end{mdframed}

\begin{mdframed}[style=MyFrame2]
    \begin{example}
    \end{example}
    \cusTi{Potencial de un dipolo}
    \begin{formatI}
        Calcular el potencial eléctrico en los puntos $\vec{x}_1$ y $\vec{x}_2$ que es generado por un dipolo de cargas puntuales $q_1=-q_2=Q$.
    \end{formatI}
    \begin{center}
        \def\svgwidth{0.7\linewidth}
        \input{./images/pot-dipolo.pdf_tex}
    \end{center}
    En el punto $\vec{x}_1$ el potencial es:
    \begin{equation*}
        \Delta \voltage = \cstcoulomb \inParentheses{\frac{q_1}{\sqrt{R^2 + y_0^2}} + \frac{q_2}{\sqrt{R^2 + y_0^2}}} = 0
    \end{equation*}

    En el punto $\vec{x}_2$ el potencial es:
    \begin{equation*}
        \Delta \voltage = \cstcoulomb \inParentheses{\frac{q_1}{x_0 + R} + \frac{q_2}{x_0 - R}}
    \end{equation*}
\end{mdframed}

\begin{mdframed}[style=MyFrame2]
    \begin{example}
    \end{example}
    \cusTi{Potencial de una barra}
    \begin{formatI}
        Calcular el potencial eléctrico en el punto $\vec{x}_0$ generado por una barra con densidad de carga lineal uniforme.
    \end{formatI}
    \begin{center}
        \def\svgwidth{\linewidth}
        \input{./images/pot-barra.pdf_tex}
    \end{center}
    Para una distribución contínua, el potencial es el aporte de cada diferencial de carga:
    \begin{equation*}
        \Delta \voltage = \cstcoulomb \int \frac{\dif q}{r}
    \end{equation*}
    Donde:
    \begin{equation*}
        \left\{
        \begin{aligned}
            \dif q &= \lambda \, \dif x
            \\
            r &= \sqrt{x^2+H^2}
        \end{aligned}
        \right.
    \end{equation*}
    Luego:
    \begin{equation*}
        \Delta \voltage = \cstcoulomb \, \lambda \int_D^{D+L} \frac{1}{\sqrt{x^2+H^2}} \, \dif x
    \end{equation*}
\end{mdframed}

\begin{mdframed}[style=MyFrame2]
    \begin{example}
    \end{example}
    \cusTi{Potencial de un anillo}
    \begin{formatI}
        Calcular el potencial eléctrico en el punto $\vec{x}_0$ generado por un anillo aislante con carga $Q$ uniformemente distribuída.
    \end{formatI}
    \begin{center}
        \def\svgwidth{\linewidth}
        \input{./images/pot-anillo.pdf_tex}
    \end{center}
    Para una distribución contínua, el potencial es el aporte de cada diferencial de carga:
    \begin{equation*}
        \Delta \voltage = \cstcoulomb \int \frac{\dif q}{r}
    \end{equation*}
    Donde:
    \begin{equation*}
        r = \sqrt{D^2 + R_0^2}
    \end{equation*}
    Luego:
    \begin{equation*}
        \Delta \voltage = \frac{\cstcoulomb}{\sqrt{D^2 + R_0^2}} \int \dif q = \frac{\cstcoulomb \, Q}{\sqrt{D^2 + R_0^2}}
    \end{equation*}
\end{mdframed}

\begin{mdframed}[style=MyFrame2]
    \begin{example}
    \end{example}
    \cusTi{Potencial de un disco}
    \begin{formatI}
        Calcular el potencial eléctrico en el punto $\vec{x}_0$ generado por un disco con densidad de carga superficial uniforme.
    \end{formatI}
    \begin{center}
        \def\svgwidth{\linewidth}
        \input{./images/pot-disco.pdf_tex}
    \end{center}
    Para una distribución contínua, el potencial es el aporte de cada diferencial de carga:
    \begin{equation*}
        \Delta \voltage = \cstcoulomb \int \frac{\dif q}{r}
    \end{equation*}
    Donde:
    \begin{equation*}
        \left\{
        \begin{aligned}
            \dif q &= \sigma \, \dif y \, \dif z = \sigma \, R \, \dif R \, \dif \phi
            \\
            r &= \sqrt{D^2+R^2}
        \end{aligned}
        \right.
    \end{equation*}
    Luego:
    \begin{align*}
        \Delta \voltage &= \cstcoulomb \, \sigma \, \int_0^{2\pi} \dif \phi \int_0^{R_0} \frac{R}{\sqrt{D^2+R^2}} \, \dif R
        \\[1ex]
        &= 2 \, \pi \, \cstcoulomb \, \sigma \inParentheses{\sqrt{R_0^2 + D^2} - D}
    \end{align*}
\end{mdframed}


\subsection{Generador Van de Graaff}

\begin{center}
    \def\svgwidth{0.8\linewidth}
    \input{./images/pot-van-de-graaff.pdf_tex}
\end{center}


\subsection{Relación entre campo y potencial}

La diferencia de potencial eléctrico, es conocida como tensión eléctrica, voltaje, potencial eléctrico o simplemente potencial.

De la definición \ref{defn:potEle} se deduce que el diferencial de potencial eléctrico es
\begin{equation*}
    \dif \voltage = - \Vec{E} \, \dif \Vec{s}
\end{equation*}

Lo cual implica $\Vec{E} = - \grad \voltage$ que en una dimensión es:
\begin{equation*}
    E(x) = - \frac{\dif}{\dif x} \voltage(x)
\end{equation*}

La siguiente imagen representa una esfera conductora con carga positiva.
Se observa que a medida que aumenta la distancia $(r)$, el campo $(E)$ disminuye.
La gráfica de $\voltage(r)$ es decreciente, y siempre tiene pendiente negativa.
La pendiente de $\voltage(r)$ es la derivada con respecto a la distancia que, por definición, es el campo eléctrico.

\begin{center}
    \def\svgwidth{0.8\linewidth}
    \input{./images/pot-ele-esfera-cond.pdf_tex}
\end{center}

El signo negativo se deduce por ser el campo eléctrico positivo si $r > R$ y negativo si $r < -R$ y la pendiente de $\voltage(r)$ negativa y positiva respectivamente.
Además, se puede observar que si la distancia es $-R < r < R$, el potencial eléctrico $\voltage(r)$ es constante con lo cual el campo es nulo ya que es la derivada con respecto de la distancia.


\section{Capacitancia}

Un recipiente con un volumen $(V)$ mayor va a tener más capacidad de almacenar gas.
Esto va a tener ciertas implicancias sobre la masa $(m)$ y la presión $(P)$.
\begin{equation*}
    V_1 > V_2 \Rightarrow
    \left\{
    \begin{aligned}
        m_1 = m_2 & \Rightarrow P_1 < P_2
        \\
        P_1 = P_2 & \Rightarrow m_1 > m_2
    \end{aligned}
    \right.
\end{equation*}

Si se define la capacidad del recipiente como $C= \sfrac{m}{P}$, a partir de las implicaciones anteriores se puede concluir que el recipiente de mayor volumen es efectivamente el de mayor capacidad.
\begin{gather*}
    \left\{
    \begin{aligned}
        P_1 < P_2 & \Rightarrow \frac{m}{P_1} > \frac{m}{P_2} \Rightarrow C_1 > C_2
        \\[1ex]
        m_1 > m_2 & \Rightarrow \frac{m_1}{P} > \frac{m_2}{P} \Rightarrow C_1 > C_2
    \end{aligned}
    \right.
    \\[1em]
    C_1 > C_2 \Rightarrow V_1 > V_2
\end{gather*}

La capacitancia o capacidad eléctrica es la cantidad de carga $(q)$ por unidad de tensión $(\Delta \voltage)$ que un capacitor o condensador puede almacenar.

\begin{mdframed}[style=MyFrame1]
    \begin{defn}
    \end{defn}
    \cusTi{Capacitancia}
    \begin{equation*}
        C = \frac{q}{\Delta \voltage}
    \end{equation*}
\end{mdframed}

\begin{mdframed}[style=MyFrame2]
    \begin{example}
    \end{example}
    \cusTi{Capacitor de placas}
    \begin{formatI}
        Capacitancia que generan dos placas paralelas, considerando que el campo eléctrico entre ellas es uniforme.
    \end{formatI}
    La capacidad está dada por
    \begin{equation*}
        C = \frac{q}{E \, \Delta x} = \frac{q}{\frac{\sigma}{\epsilon_0}\Delta x} = \frac{q}{\tfrac{q}{A \, \epsilon_0}\Delta x}
    \end{equation*}

    Obteniendo así
    \begin{equation*}
        C = \frac{\epsilon_0 A}{\Delta x}
    \end{equation*}
\end{mdframed}

\begin{mdframed}[style=MyFrame2]
    \begin{example}
    \end{example}
    \cusTi{Capacitor esférico}
    \begin{formatI}
        Capacitancia que generan o bien dos esferas concéntricas, o bien una única esfera como un caso particular en la que la externa tiene radio infinito.
    \end{formatI}
    El potencial eléctrico entre las esferas (en valor absoluto) está dado por:
    \begin{align*}
        \Delta \voltage &= \int_C \vec{E} \cdot \dif \vec{s}
        \\[1ex]
        &= \int_{r_1}^{r_2} \frac{\cstcoulomb \, q}{r^2} \dif r
        \\
        &= \cstcoulomb \, q \inParentheses{\frac{1}{r_1} - \frac{1}{r_2}}
    \end{align*}

    Así:
    \begin{align*}
        C &= \frac{q}{\Delta \voltage}
        \\
        &= \frac{1}{\cstcoulomb \inParentheses{\frac{1}{r_1} - \frac{1}{r_2}}}
    \end{align*}

    Obteniendo, para dos esferas concéntricas
    \begin{equation*}
        C = \frac{r_1 \, r_2}{\cstcoulomb \inParentheses{r_2 - r_1}}
    \end{equation*}

    O bien, si $r_2 \to \infty$, se tiene para una única esfera
    \begin{equation*}
        C = \frac{r_1}{\cstcoulomb} = 4 \, \pi \, \epsilon_0 \, r_1
    \end{equation*}
\end{mdframed}

Para calcular la energía almacenada en un capacitor se hace:
\begin{gather*}
    \frac{\dif W}{\dif q} = \frac{\dif q}{\dif q} \, \Delta \voltage = \Delta \voltage
    \\
    \dif W = \Delta \voltage \, \dif q = \frac{q}{C} \, \dif q
    \\
    \int \dif W = \int \frac{q}{C} \, \dif q
    \\
    W = \barrow{\frac{q^2}{2C}}{0}{q_1}
    \\
    \sub{E}{pot} = \frac{q^2}{2C} = \frac{q \, \Delta \voltage}{2}
\end{gather*}

\begin{mdframed}[style=MyFrame1]
    \begin{prop}
    \end{prop}
    \cusTi{Energía almacenada en un capacitor}
    \begin{equation*}
        \sub{E}{pot} = \frac{C \inParentheses{\Delta \voltage}^2}{2}
    \end{equation*}
\end{mdframed}

\begin{mdframed}[style=MyFrame2]
    \begin{example}
    \end{example}
    \cusTi{Energía almacenada en placas paralelas}

    El potencial eléctrico es
    \begin{equation*}
        \Delta \voltage = E \, \Delta x
    \end{equation*}
    y la capacitancia
    \begin{equation*}
        C = \tfrac{\epsilon_0 \, A}{\Delta x}
    \end{equation*}
    obteniendo
    \begin{equation*}
        \sub{E}{pot} = \tfrac{1}{2} \, \tfrac{\epsilon_0 \, A}{\Delta x} \inParentheses{E \, \Delta x}^2
    \end{equation*}
    y si $V_0 = A \, \Delta x$ es el volumen se tiene
    \begin{equation*}
        \frac{\sub{E}{pot}}{V_0} = \frac{\epsilon_0 \, E^2}{2}
    \end{equation*}
\end{mdframed}

El campo eléctrico máximo que se puede dar en un capacitor está dado por la resistencia dieléctrica.
Si la magnitud del campo es mayor que esta, esto es, entonces el dieléctrico pasa a ser conductor.
Por lo tanto, para un dieléctrico se cumple:
\begin{equation*}
    E < \epsilon
\end{equation*}

La capacitancia está definida para el vacío.
Si entre los conductores de un capacitor en vez de vacío hay un material dieléctrico, la tensión va a estar dada por
\begin{equation*}
    \Delta \voltage = \frac{\Delta V_0}{k}
\end{equation*}

Quedando la capacitancia:
\begin{equation*}
    C = k \, C_0
\end{equation*}